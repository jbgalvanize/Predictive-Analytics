\documentclass[11pt]{article}
\setlength{\topmargin}{-.67 in} \setlength{\textheight}{10.0 in}
\setlength{\leftmargin}{-0in} \setlength{\textwidth}{7 in}
\setlength{\evensidemargin}{-0.25 in}
\setlength{\oddsidemargin}{-.25 in}
%\setlength{\parindent}{0.0 in}


\begin{document}
\begin{center}
\LARGE \textbf{BANA 6640} \\
\Large \textbf{Decision Analysis} \\
\normalsize Section 001 Spring 2015\\
\end{center}
\normalsize
\begin{tabbing}
\textbf{Instructor}: \qquad \qquad \qquad \= Josh Bernhard  \qquad \qquad \qquad \qquad \qquad \= Office: BUS 5016\\
\> Phone: (XXX) XXX-XXXX \>  Email: joshua.bernhard@ucdenver.edu\\[.1cm]
\textbf{Office Hours:} \> By appointment\\[.1cm]
\textbf{Lecture:}  \> T 6:30-9:15PM, BUS 3007 \> \\[.1cm]
\textbf{Course web page:} \>   Canvas \\[.1cm]
\end{tabbing}


\vspace*{.3cm}
\noindent\textbf{{\large Course Description}}
\begin{enumerate}
\item[] Examines business decision making under conditions of risk and uncertainty
using quantitative decision analysis methods such as utility theory, value of
information, decisions with conflicting objectives and hierarchical structured
models. Applications include decisions commonly encountered in capital acquisitions,
financial investments, quality control, project selection, strategic planning, production control and human resource management. Student computer-assisted projects
are conducted. \\[-.6cm]
\end{enumerate}


\vspace*{.3cm}
\noindent\textbf{{\large Required Materials}}
\begin{enumerate}
    \item  \textbf{You may want access to} \emph{Making Hard Decisions: An Introduction to Decision Analysis}, 
    Clemen, Any Edition (But this is no longer required).\vspace{-2.5mm}
  	\item  R Statistical Software Access \vspace{-2.5mm}
    \item  Regular Computer Access - it is ideal to bring a computer to class.
\end{enumerate}

\vspace*{.3cm}
%\noindent\textbf{{\large Exam Schedule with material (approximate)}}\\[-.4cm]
%\begin{center}
%    \begin{tabular}{lll}
%        Exam 1 & Wednesday, February 27th & 1-4, 12-14.1  \\
%        Exam 2 & Wednesday, April 17th & 1-4, 12-14.1, 15-16, 6 \\
%        Final Exam$^*$$^*$ & TBA (sometime during week of May 13) &  Cumulative 1-4, 6, 12-16, 19, 21-22\\
%    \end{tabular}\\
%    \vspace{.5cm}
%
%    $^*$$^*$The Final Exam is \emph{comprehensive}. Our Final Exam is
%    \textbf{tentatively} scheduled from xx am to xx am on xxxx, May xx, 2013.
%    \textbf{The Final Exam time and date are tentative until finalized by the
%    University. }
%\end{center}

%\vspace*{.3cm}

\noindent\textbf{{\large Grading}}
\vspace*{-.1cm}
\[\mbox{Final letter grades (on plus/minus scale) are determined from:\,\,\, Final Numeric Grade} = \]
\[0.20\times\mbox{HW\%} + 0.20\times\mbox{Quiz\%} + 0.30\times\mbox{Exam1\%} + 0.30\times\mbox{Exam2\%}\]

\vspace*{.3cm}

\textbf{{\large Extra Credit:}} This class does not offer any extra credit opportunities throughout the semester. \\

\textbf{{\large Grade Computation:}} Your grade will not be determined on a scale any stricter than a standard 90-80-70 scale, where '+' and '-' grades are given at each 3.33\% cutoff. \\


\vspace*{.5cm}

\noindent\textbf{{\large Course Information and Policies}}

\begin{itemize} \itemsep 0.1in
\item[]  \textbf{Reading assignments} The textbook for this course is optional.  I feel that the lecture notes, labs, homework assignments, and in class activities will provide all materials necessary to be successful in this course.  However, you may feel that another perspective is necessary to fully grasp the course material. I will not assign any problems from the book, but I will provide the approximate chapters we are covering for you to read if you would like to follow along.  
\begin{itemize} \itemsep 0.2cm
\item \emph{The end of each chapter also has an exercise
set. These are also good review for midterm exams as well as the
final exam.} Exercises should be attempted as topics are covered
rather than waiting until the end of the week (or month or
semester!). Your efforts at working exercises will enhance
your performance on the quizzes, homework projects, and exams.\\
\end{itemize}

\item[] \textbf{Homework Assignments} will be assigned,
collected, and graded a few times during the semester.
\begin{itemize} \itemsep 0.2cm
\item Homework assignments will be posted on Canvas.
\item Instructions specific to each assignment will be discussed in class when the assignment is announced.
\item \textbf{Late submissions will not be accepted.}
\end{itemize}

\item[] \textbf{Quizzes} will be assigned,
collected and graded on some weeks during the semester.
\begin{itemize} \itemsep 0.2cm
\item Assignments will be due by the end of class.
\item Quizzes will be posted on Canvas along with solutions following the completion of quiz.
\item Instructions specific to each quiz will be discussed at the beginning of each quiz.
\end{itemize}

\item[] \textbf{R} (pronounced ``R") is statistical software, which will be used on the homework.  You are responsible for knowing how to use R.  The R software is installed in many computer labs on campus (I will try to get a specific list) for your access. Instructions to download R on your personal computer (at no cost) can also be found, and I will send an email to the link.\\


\item[] \textbf{Office Hours} If you have any issues with installing R, I will be happy to assist you. For assistance feel free to see me during office hours.  I am willing to help with any questions related to the lecture material, homework problems and R. Questions are welcome but serious effort is expected before seeking help.  I am also available quickly via email.  \\

\item[] There are two \textbf{Exams} for this class.  For the exam you will be allowed a computer to conduct calculations using $R$. A schedule providing the dates for the exams (as well as homework due dates) is provided in Canvas.
\begin{itemize} \itemsep 0.2cm
\item Decision Analysis has two components: \textbf{calculation} and \textbf{interpretation}, and you are responsible for both components on exams. In
order to receive full credit for a calculation, one must show the
work that led to the calculation \emph{and} have the correct answer.
In order to receive full credit for an interpretation, one must use
complete sentences, proper grammar, and correct spellings of words
in order to clearly communicate thoughts.  Approximately half of all
half points are calculation-based and half are interpretation-based.  Tables and graphs are highly encouraged in conveying your arguments.
\end{itemize}



\item[] \textbf{Course Page in Canvas}: 
\begin{center}
\texttt{ BANA 6640 001 Decision Analysis 2014 Fall}
\end{center}

\noindent Important information regarding the course (such as homework, exams, in class activities, etc.) can be found there. Please check for updates
frequently. \\[.15cm]


\item[] \textbf{{\large Miscellaneous}}

\begin{itemize} \itemsep .1in
\item  An appropriate \textbf{calculator} for this
class should do two-variable statistics (and the financial functions
you will need later in your business program).  Programmable calculators may be used, but any
efforts to program formulas, text, etc.  into them for use on an
exam will be considered academically dishonest and treated according
to university policy.  \emph{The instructor reserves the right to clear the memory
of selected calculators during exam times.}

\item  Your \textbf{attendance} to each class
(listening) and your genuine effort on the homework projects,
quizzes, and exams (doing) are two of the most important factors
affecting your success in this class. Feel free to ask me about problems after you
have honestly tried---\emph{my office hours are for you}.

NA

\item \textbf{Conduct}. I expect strict adherence to the code of conduct as stated in the Catalog of the University of Colorado at Denver.  I will take action and initiate due process against violations of the code of conduct.  Violations include verbal and physical disruption or intimidation.  Actions to initiate due process can include a formal report to the Behavioral Assessment and Threat Assessment Team (BETA) or the Internal Affairs Committee of the Business School at the University of Colorado at Denver.  Conduct also includes etiquette before, during, and after each class that is respectful to all people, the facilities, and the institution.   

\item \textbf{Commitment Expectation}. Faculty policies state that courses in the business school should require of their students a workload in and out of class that enhances quality education and is consistent with the amount of academic credit for a course.  It is expected that out of class time a student will spend is from two to three times the time spent in class.  This course requires homework and exams due on a regular basis.  

\item \textbf{Disability-Accommodation}. University of Colorado complies with the
Americans with Disabilities Act and Sect 504 of the Rehabilitation
Act.  If you have a documented disability and anticipate needing
accommodations in this course, please contact me within the first
two weeks of the semester.  Retroactive requests for accommodations
will not be honored.  You will need adequate documentation as sufficient proof to receive necessary accommodations.
\end{itemize}
\end{itemize}
%\newpage
%\begin{center}
%\textsc{tentative schedule of topics to be covered -- Spring 2013}
%\end{center}

%All sections should cover the same topics in the same order
%at roughly the same time. We may have to make some adjustments
%for all sections to stay on the same pace. However, this is the pace we will try
%to keep (approximately).\\
%\\

%\noindent \textbf{Homework due dates are tentative and likely to change to adjust to the actual pace of the course.}


%\hspace*{-4cm}\\
%\begin{center} \footnotesize
%\begin{tabular}{|c|c|c|l|c|}
%  \hline
  % after \\: \hline or \cline{col1-col2} \cline{col3-col4} ...
%  \textbf{Week} & \textbf{Date} & \textbf{Chapter} & \textbf{Topics} & \textbf{Homework}\\
%  \hline
%1 & 01/23 & 1-3 & \textsc{introduction (what is statistics?), data \& data types,} & \\		
%   & 01/23 &	      &	 \textsc{graphical displays for categorical data} &\\	 \hline			
%2 & 01/28 &	4    &	 \textsc{graphical displays and numerical summaries} & \textsc{homework} 1\\		
%   & 01/30 &	      &	 \textsc{for numerical data} & \\ \hline		
%3 & 02/04 &	12  &	 \textsc{the normal probability model,} & \textsc{homework} 2 \\		
%   & 02/06 &	 &	 \textsc{normal random variable} 	&	\\ \hline		
%4 & 02/11 &	12 &	 \textsc{``backward'' normal calculations, } 	& \textsc{homework} 3\\		
%   & 02/13 &	     &	 \textsc{assessing normality} &	\\ \hline	
%5 & 02/18 &	13 &	\textsc{properties of sampling, sampling variation}  &  \textsc{homework} 4\\	
%   & 02/20 &	12 &	\textsc{modeling sampling variation, using a normal model} &	\\ \hline	
% & 02/25 &	14.1&	\textsc{central limit theorem, sampling distribution of} &\\ 
%6  &           & & \textsc{the sample mean} & \\
%  & 02/27 &	      &	 \textbf{Exam 1} (Wednesday) 	&	Exam1 HW \\ \hline		
%7 & 03/04 &	15.1 &  \textsc{ranges for parameters, proportions} &	\\ 	    
%   & 03/06 & 15.2, 15.3 & $t$-\textsc{distribution}, \textsc{confidence interval for the mean} & \\ \hline
%8 & 03/11 &	15.4 & \textsc{interpreting confidence intervals,}& \textsc{homework} 5 \\		
%   & 03/13 &	15.5 & \textsc{manipulating confidence intervals, margin of error}	&	 \\ \hline		
%9 & 03/18 &	16.1 & \textsc{concepts of statistical test} & \textsc{homework} 6\\		
%   & 03/20 &	16.3 & \textsc{testing the mean}	&	 \\ \hline	
%& 03/25-03/27	&		&	 \textbf{Spring Break}	&	 \\ \hline	
%& 04/01 &	 16.4 & \textsc{practical vs statistical significance} 	&  \\		
%10    & 04/03 & 6.1--6.3 & \textsc{scatterplots, association in scatterplots}	& \textsc{homework} 7 \\ 
%    &           &  & \textsc{measuring association} & \\ \hline	
%11& 04/08 & 6.5, 19.1 & \textsc{spurious correlation, fitting a line to data} & \textsc{homework} 8	 \\ 
%    & 04/10 &  19.2 & \textsc{interpreting the fitted line} & \\ \hline	
%12& 04/15 &	  & \textsc{practice}	&	\\		
%    & 04/17 &    & \textbf{Exam 2} (Wednesday) &	Exam2 HW \\ \hline		
%13& 04/22 &	19.3 &\textsc{predictions, properties of residuals}& \\	
%    & 04/24 & 19.4, 19.5   & \textsc{explaining variation, conditions for simple regression} & \\ \hline			
%  \multicolumn{3}{|c|}{11/19-11/23} & \multicolumn{2}{c|}{\textbf{Thanksgiving Break}} \\ \hline	
%    & 04/29 &	21.1 &	\textsc{the simple linear regression model (srm)} &\\		
%14 &           & 21.2 & \textsc{conditions for srm} &  \textsc{homework} 9 \\
%    & 05/01 &  21.3  &	\textsc{inference in Regression} &  \\ \hline			
%15& 05/06 &	22&	\textsc{regression diagnostics}	&		\\	
%    & 05/08 &   &	\textsc{leverage points,  outliers} &	\textsc{homework} 10\\ \hline		
%	\hline
%\end{tabular}
%\end{center}
%\vfill
%\begin{center}
This syllabus is subject to change.
\end{document}
